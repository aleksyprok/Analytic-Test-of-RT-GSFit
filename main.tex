\documentclass{article}
\usepackage{amsmath}
\usepackage[a4paper, margin=1in]{geometry}

\title{Analytic Test of RT-GSFit}
\author{Alexander Prokopyszyn}
\date{May 2025}

\begin{document}

\maketitle

\section{Deriving Bessel's Eqn from the Grad-Shafranov Eqn}

The Grad-Shafranov equation is given by
\[\frac{\partial^2 \psi}{\partial R^2} - \frac{1}{R}\frac{\partial \psi}{\partial R} + \frac{\partial^2 \psi}{\partial z^2} = -\mu_oR^2p'(\psi) - F(\psi)F'(\psi).\]
We will assume a plasma beta, $\beta=0$. We are interested in solving the Grad-Shafranov equation inside the LCFS. 
% Let $R=R_o$, $z=z_o$ be the coordinates of the o-point. 
% We will assume that
% \[\frac{|R-R_o|}{R_o} = O(\epsilon),\]
% \[R\frac{\partial}{\partial R} = O\left(\frac{1}{\epsilon}\right)\]
% \[R\frac{\partial}{\partial z} = O\left(\frac{1}{\epsilon}\right)\]
% \[\frac{|R-R_o|}{R_o} \ll 1,\]
% \[\implies \frac{1}{R} = \frac{1}{R-R_0 + R_0} = \frac{1}{R_0}\left(\frac{1}{1+\frac{R-R_0}{R}}\right)\]
We will take the large aspect-ratio limit and assume $R\rightarrow \infty$.
Hence, we can simplify the Grad-Shafranov to
\[\frac{\partial^2 \psi}{\partial R^2} + \frac{\partial^2 \psi}{\partial z^2} = -F(\psi)F'(\psi).\]
We will assume poloidal symmetry, and so we assume $\psi=\psi(\rho)$.
Let
\[\rho = \sqrt{(R-R_o)^2+(z-z_o)^2}.\]
\[R-R_o = \rho\cos(\theta),\]
\[z-z_o = \rho \sin(\theta),\]
% \[\theta = \tan^{-1}(z-z_o, R-R_o)\]
\[\mathbf{\hat{\rho}} = \cos(\theta) \mathbf{\hat{R}} + \sin(\theta)\mathbf{\hat{z}},\]
\[\mathbf{\hat{\theta}} = -\sin(\theta) \mathbf{\hat{R}} + \cos(\theta)\mathbf{\hat{z}}.\]
Note that
\[\begin{aligned}
\frac{\partial \psi}{\partial R} &= \frac{\partial \rho}{\partial R}\frac{d \psi}{d \rho} \\
&= \frac{(R-R_o)}{\sqrt{(R-R_o)^2+(z-z_o)^2}}\frac{d \psi}{d \rho} \\
&= \cos(\theta)\frac{d \psi}{d \rho},
\end{aligned}\]
\[\begin{aligned}
\frac{\partial \psi}{\partial z} &= \frac{\partial \rho}{\partial z}\frac{d \psi}{d \rho} \\
&= \frac{(z-z_o)}{\sqrt{(R-R_o)^2+(z-z_o)^2}}\frac{d \psi}{d \rho} \\
&= \sin(\theta)\frac{d \psi}{d \rho},
\end{aligned}\]
\[\begin{aligned}
    \frac{\partial \psi}{\partial R}\mathbf{\hat{R}} + \frac{\partial \psi}{\partial z}\mathbf{\hat{z}} &= \frac{d \psi}{d \rho}(\cos(\theta) \mathbf{\hat{R}} + \sin(\theta)\mathbf{\hat{z}}) \\
    &= \frac{d \psi}{d \rho}\mathbf{\hat{\rho}},
\end{aligned}
\]
\[\begin{aligned}
\mathbf{\hat{R}}\cdot\frac{\partial \mathbf{\hat{\rho}}}{\partial R} &= \frac{\partial}{\partial R}\left(\frac{R-R_o}{\sqrt{(R-R_o^2)+(z-z_o^2)}}\right)\\
&= \frac{\sin^2\theta}{\rho}
\end{aligned}\]
\[\begin{aligned}
\mathbf{\hat{z}}\cdot\frac{\partial \mathbf{\hat{\rho}}}{\partial z} &= \frac{\partial}{\partial z}\left(\frac{z-R_o}{\sqrt{(R-R_o^2)+(z-z_o^2)}}\right)\\
&= \frac{\cos^2\theta}{\rho}
\end{aligned}\]
\[\begin{aligned}
\frac{\partial^2\psi}{\partial R^2} + \frac{\partial^2 \psi}{\partial z^2} &= \left(\mathbf{\hat{R}}\frac{\partial }{\partial R} + \mathbf{\hat{z}}\frac{\partial }{\partial z}\right)
\cdot\left(\frac{\partial \psi}{\partial R}\mathbf{\hat{R}} + \frac{\partial \psi}{\partial z}\mathbf{\hat{z}}\right) \\ 
&= \left(\mathbf{\hat{R}}\frac{\partial }{\partial R} + \mathbf{\hat{z}}\frac{\partial }{\partial z}\right) \cdot \left(\frac{d \psi}{d \rho}\mathbf{\hat{\rho}}\right) \\
&= \cos(\theta)\frac{\partial}{\partial R}\left(\frac{d \psi}{d \rho}\right)
+ \frac{d \psi}{d \rho}\mathbf{\hat{R}}\cdot\frac{\partial \mathbf{\hat{\rho}}}{\partial R}
+ \sin(\theta)\frac{\partial}{\partial z}\left(\frac{d \psi}{d \rho}\right)
+ \frac{d \psi}{d \rho}\mathbf{\hat{z}}\cdot\frac{\partial \mathbf{\hat{\rho}}}{\partial z} \\
&= \frac{d^2\psi}{d \rho^2} + \frac{1}{\rho}\frac{d \psi}{d \rho}.
\end{aligned}\]
Hence,
we can rewrite the Grad-Shafranov equation as
\[\frac{d^2\psi}{d \rho^2} + \frac{1}{\rho}\frac{d \psi}{d \rho} = -F(\psi)F'(\psi).\]

RTGSFit assumes $FF'$ is of the form
\[FF' = a(1-\psi_N),\]
where $a$ is a constant and
\[\psi_N = \frac{\psi_o-\psi}{\psi_o-\psi_b},\]
where $\psi_b$ is $\psi$ at the last closed flux surface and $\psi_o$ is the value of $\psi$ at the o-point.
Let
\[\psi(\rho)=\psi'(\rho)+\psi_b,\]
\[a'=\frac{a}{(\psi_o - \psi_b)},\]
hence by substituting our expression for $\psi$ in terms of $\psi'$ and $a$ in terms of a' we get
\[\frac{d^2\psi'}{d \rho^2} + \frac{1}{\rho}\frac{d \psi'}{d \rho} = -a'\psi'.\]
Multiplying through by $\rho^2$ and rearranging we arrive at an equation that is nearly Bessel's equation of order 0
\[\rho^2\frac{d^2\psi'}{d \rho^2} + \rho\frac{d \psi'}{d \rho} + a'\rho^2\psi'= 0.\]
Let 
\[s=\frac{a'}{|a'|}\sqrt{|a'|}\rho,\]
\[\implies \rho=\frac{a'}{|a'|}\frac{1}{\sqrt{|a'|}}s\]
\[\implies \frac{\partial }{\partial \rho} =\frac{a'}{|a'|}\sqrt{|a'|}\frac{\partial }{\partial s},\]
\[\implies \frac{\partial^2 }{\partial \rho^2} =a'\frac{\partial^2 }{\partial s^2},\]
\[\implies \rho^2 = \frac{s^2}{a'}\]
Hence, the Grad-Shafranov equation becomes
\[s^2\frac{\partial^2\psi'}{\partial s^2} + s\frac{\partial \psi'}{\partial s} + s^2\psi'= 0,\]
which is exactly of the same form as Bessel's equation of order 0.

\section{Solving Bessels' equation}

We know that $\psi'$ is of the form
\[\psi' = C_1J_0(s) + C_2Y_0(s),\]
where $J_0$ and $Y_0$ are Bessel function of order zero of the first and second kind respectively. We know that $\psi$ is finite at $R=R_o$, $z=z_o$, hence, $C_2=0$.
Past the last closed flux surface, there is no plasma current. 
Hence, the full solution for $\psi'$ is
\[\psi'(\rho) = C_1J_0(\sqrt{a'}\rho).\]
% Note that RT-GSFit assumes $\psi$ is a monotonically decreasing function away from the o-point inside the LCFS, but $J_0(\sqrt{a'}\rho)$ oscillates. This means the corodinate of the LCFS, $\rho=\rho_b$, must be less than the value of $\rho$ where $J_0(\sqrt{a'}\rho)$ has its first turning point (besides the turning point at $\rho=0$). Note that
% \[\frac{d }{d x}J_0(x) = -J_1(x),\]
% and $J_1(x)$ has its first root at about $x=\pm3.8317$ (besides the root at $x=0$). Hence, we require
% \[\sqrt{a'}\rho < 3.8317 \implies \rho < \frac{3.8317}{\sqrt{a'}}.\]
% Note that
% \[a = \frac{\psi_o-\psi_b}{\rho_b^2}.\]
% we assume that $\psi_o>\psi_b  a>0$.

\section{Tidying up solution}

We know that
\[\psi = C_1J_0(\sqrt{a'}\rho) + \psi_b,\]
which we can rewrite as
\[\psi = C_1J_0\left(a^*\frac{\rho}{\rho_b}\right) + \psi_b,\]
where
\[a^* = \sqrt{\frac{a\rho_b^2}{\psi_o-\psi_b}}.\]
We know that $\psi(\rho_b)=\psi_b$, hence,
\[ C_1J_0(a^*)=0.\]
We will assume that we have chosen $a$ such that $a^*$ is a zero of $J_0$.
We know that $\psi(0) = \psi_o$, hence
\[\psi = (\psi_o - \psi_b)J_0\left(a^*\frac{\rho}{\rho_b}\right) + \psi_b,\]


% Let us choose $a$ such that
% \[J_0(\sqrt{a'}\rho_b)=0,\]
% Specifically, the first positive zero of $J_0$.
% Note that
% \[a' = \frac{a}{\psi_o-\psi_b}.\]
% Let
% \[a^* = \sqrt{\frac{a\rho_b^2}{\psi_o-\psi_b}},\]
% \[\implies J_0(\sqrt{a'}\rho) = J_0\left(a^*\frac{\rho}{\rho_b}\right),\]
% hence, $a^*$ needs to be the coordinate of the first positive zero of $J_0$ and
% \[a = \frac{\psi_o - \psi_b}{\rho_b^2}(a^*)^2,\]
% \[\psi = (\psi_o-\psi_b)J_0\left(a^*\frac{\rho}{\rho_b}\right) + \psi_b\]

% \[a = \frac{\psi_o}{\rho_b^2}(a^*)^2\]

% Without loss of generality, we will set $\psi_b=0$. Hence
% \[\psi = \psi_oJ_0\left(\sqrt{\frac{a}{\psi_o}}\rho\right).\]
% Let
% \[a = \frac{\psi_o}{\rho_b^2}(a^*)^2,\]
% where $a^*>0$,
% \[\implies a^* = \sqrt{\frac{a\rho_b^2}{\psi_o}}.\]
% Hence,
% \[\psi = \psi_oJ_0\left(a^*\frac{\rho}{\rho_b}\right).\]
% Since $\psi(\rho_b)=\psi_b=0$ we know that
% \[J_0(a^*)=0,\]
% \[\implies a^* \approx 2.404825557695772768621631879326454643124244909146\]

\section{Solution for $\rho>\rho_b$}

For $\rho>\rho_b$ there is no plasma current and the Grad-Shafranov equation is given by
\[\rho^2\frac{d^2\psi}{d \rho^2} + \rho\frac{d \psi}{d \rho} = 0\]
The equation has the solution
\[\psi = C_1\ln\left(\frac{\rho}{\rho_b}\right) + \psi_b,\]
and satisfies continuity at $\rho=\rho_b$.
We also require continuity of the derivative of $\psi$, note that
\[\frac{d}{d\rho}\left[(\psi_o-\psi_b)J_0\left(a^*\frac{\rho}{\rho_b}\right)\right]=-\frac{(\psi_o-\psi_b)a^*}{\rho_b}J_1\left(a^*\frac{\rho}{\rho_b}\right),\]
\[\frac{d}{d\rho} C_1\ln\left(\frac{\rho}{\rho_b}\right) = \frac{C_1}{\rho},\]
hence,
\[\frac{C_1}{\rho_b} = -\frac{(\psi_o-\psi_b)a^*}{\rho_b}J_1(a^*),\]
\[\implies C_1 = -(\psi_o-\psi_b)a^*J_1(a^*),\]
% Note that
% \[J_1(a^*) \approx 0.519147497289466788140202640208624244569819046868.\]
Hence, for $\rho\ge\rho_b$
\[\psi = -(\psi_o-\psi_b)a^*J_1(a^*)\ln\left(\frac{\rho}{\rho_b}\right) + \psi_b.\]

\section{Total current}

For $\rho>\rho_b$, the poloidal field is given by
\[\begin{aligned}
B_\theta &= \frac{1}{R}\frac{d\psi}{d\rho}, \\
&= -\frac{(\psi_o-\psi_b)a^*J_1(a^*)}{(R_o+\rho\cos\theta)\rho}.
\end{aligned}\]
Using Ampere's law, namely,
\[2\pi\rho B_\theta = \mu_0 I_\text{total},\]
\[\begin{aligned}
\oint_{\rho=\rho_0}B_\theta \rho_0d\theta &= \frac{(\psi_o-\psi_b)a^*J_1(a^*)}{R_o}\frac{2\pi}{\sqrt{1-\frac{\rho_0^2}{R_o^2}}}\\
&= 2\pi\frac{(\psi_o-\psi_b)a^*J_1(a^*)}{R_o}\quad (\text{for } \rho_0/R_o\rightarrow0)  \\
&=\mu_0 I_\text{total}.
\end{aligned}\]
% \[\begin{aligned}
% \oint_{\rho=\rho_0}B_\theta \rho_0 d\theta &= 2\pi\rho_o\frac{(\psi_o-\psi_b)a^*J_1(a^*)}{R_o\rho_0} + \oint_{\rho=\rho_0}\frac{(\psi_o-\psi_b)a^*J_1(a^*)}{R_o\rho_0}\left(\frac{\rho\cos\theta}{R_o}\right)^2 d\theta + ...\\
% &=\mu_0 I_\text{total}.
% \end{aligned}\]
Taking the leading-order term, we know that
\[I_{total} = \frac{2\pi}{R_o\mu_0}(\psi_o-\psi_b)a^*J_1(a^*),\]
\[\implies \psi_o-\psi_b = \frac{\mu_0I_\text{total} R_o}{2\pi a^*J_1(a^*)}\]

\section{Calculating expressions for $\psi_o$, $\psi_b$}

We need to ensure that $\psi$ is zero at $(R, z) = (0,0)$. Where we assume this is outside the region $\rho<\rho_b$.
Let
\[d = \sqrt{R_o^2-z_o^2}.\]
We impose $\psi(d) = 0$. Hence
\[\psi_b = \frac{\mu_0I_\text{total}R_o}{2\pi}\ln\left(\frac{d}{\rho_b}\right),\]
\[\implies \psi_o = \frac{\mu_0I_\text{total}R_o}{2\pi}\left[\ln\left(\frac{d}{\rho_b}\right)+\frac{1}{a^*J_1(a^*)}\right]\]

\section{Full solution}

The full solution for $\psi$ is given by
\[
\psi(\rho) = \begin{cases}
\Delta_\psi J_0\left(a^*\dfrac{\rho}{\rho_b}\right) + \psi_b, &\rho\le\rho_b, \\
-\Delta_\psi a^*J_1(a^*)\ln\left(\dfrac{\rho}{\rho_b}\right) + \psi_b, & \rho\ge\rho_b,
\end{cases}
\]
where
\[\Delta_\psi = \psi_o - \psi_b\]
\[\psi_o = \frac{\mu_0}{2\pi}I_\text{total}R_o\left[\ln\left(\frac{d}{\rho_b}\right)+\frac{1}{a^*J_1(a^*)}\right],\]
\[\psi_b = \frac{\mu_0}{2\pi}I_\text{total}R_o\ln\left(\frac{d}{\rho_b}\right),\]
\[\Delta_\psi = \frac{\mu_0I_\text{total}R_o}{2\pi a^* J_1(a^*)},\]
\[\begin{aligned}
a &= \frac{(a^*)^2(\psi_o-\psi_b)}{\rho_b^2} \\
&= \frac{\mu_0 a^* I_\text{total}R_o}{2\pi J_1(a^*)\rho_b^2},
\end{aligned}\]
and $a^*$ is the coordinate of a positive zero of $J_0$. Hence, the solution is not unique but if we impose that $\psi$ needs to decrease monotonically away from $\psi_o$ then this forces $a^*$ to be the first positive zero of $J_0$.

% Now we will write our solution for $\psi$ in the form we will use in RTGSFIT. We will set $\psi = 0$ at $(R, Z) = (0, 0)$. Let
% \[d = \sqrt{R_o^2-z_o^2}.\]
% Hence, for $\rho \ge \rho_b$
% \[\begin{aligned}
% \psi &= -\psi_oa^*J_1(a^*)\left[\ln\left(\frac{\rho}{\rho_b}\right)-\ln\left(\frac{d}{\rho_b}\right)\right] \\
% &= -\psi_oa^*J_1(a^*)\ln\left(\frac{\rho}{d}\right)
% \end{aligned}\]
% and for $\rho\le\rho_b$
% \[
% \psi = \psi_oJ_0\left(a^*\frac{\rho}{\rho_b}\right) +\psi_oa^*J_1(a^*)\ln\left(\frac{d}{\rho_b}\right).
% \]
% Hence,
% \[\psi_b = \psi_oa^*J_1(a^*)\ln\left(\frac{d}{\rho_b}\right).\]

\section{Magnetic field}

The magnetic field is given by
\[\begin{aligned} 
    B_R &= B_\theta \vec{\hat{\theta}}\cdot \vec{\hat{R}} \\
    &= -\frac{1}{R}\sin(\theta) \frac{d\psi}{d\rho} \\
    &= -\frac{1}{R}\frac{z - z_o}{\rho}\frac{d\psi}{d\rho}
\end{aligned}\]
\[\begin{aligned} 
B_z &= B_\theta \vec{\hat{\theta}}\cdot \vec{\hat{z}} \\
&= \frac{1}{R}\cos(\theta)\frac{d\psi}{d\rho} \\
&= \frac{1}{R}\frac{R-R_o}{\rho}\frac{d\psi}{d\rho}
\end{aligned}\]
% \[\begin{aligned}
% B_\theta &= -\sin(\theta)B_R + \cos(\theta)B_z \\
% &= -\frac{1}{R}\frac{d\psi}{d\rho}.
% \end{aligned}\]

\end{document}
